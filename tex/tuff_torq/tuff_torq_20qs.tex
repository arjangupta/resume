\documentclass{article}

\begin{document}

\title{Tuff Torq --- Screening Questions}
\author{Candidate: Arjan Gupta}
\date{\today}
\maketitle

% Job Description :


% Responsibilities include:

% Develop and implement software solutions for off-highway autonomous vehicles, focusing on navigation, obstacle detection and avoidance, path planning, and optimization algorithms.
% Collaborate with cross-functional teams, including hardware engineers, mechanical engineers, and product managers, to define requirements, design features, and integrate software components into the autonomous control system.
% Research and evaluate emerging technologies, industry trends, and best practices in autonomous systems, robotics, and artificial intelligence to enhance the capabilities of our autonomous control system.
% Debug, troubleshoot, and resolve software defects, ensuring the reliability and stability of software.
% Conduct rigorous testing, both in simulation and real-world environments, to validate software performance and safety.
% Optimize system performance, including power management, battery life, and computational efficiency, to maximize vehicle capabilities.
% Collaborate with the firmware and embedded systems teams to ensure seamless integration of software and hardware components, enabling efficient communication and controls.
% Document software designs, specifications, and implementation details, ensuring comprehensive and up-to-date technical documentation.
 

% Qualifications:

% Bachelor's or Master's degree in Computer Science, Software Engineering, Robotics, or a related field.
% Proven experience in software development for autonomous systems, robotics, or similar domains.
% Strong proficiency in programming languages such as C++, Python, or Java.
% Familiarity with robotics frameworks, libraries, and tools (e.g., ROS, OpenCV, TensorFlow) is highly desirable.
% Experience in developing algorithms for perception, localization, mapping, and motion planning in autonomous systems.
% Solid understanding of control systems, sensor fusion, and machine learning techniques.
% Knowledge of software engineering best practices, including version control, code reviews, and software testing.
% Ability to work effectively in a collaborative, fast-paced environment with cross-functional teams.
% Excellent problem-solving skills, attention to detail, and a strong commitment to quality.
% Strong communication skills to effectively convey complex technical concepts to both technical and non-technical stakeholders.
% Must be able to lift per Tuff Torq Corporation guidelines.
% Minimal travel may be required

\begin{itemize}
    \item \textbf{Question 1:} Can you tell me about your relevant experience for this position?
    \begin{itemize}
        \item \textbf{Answer:} Absolutely. First off, I have a Bachelor's degree in 
        Computer Engineering from the University of Kansas, and 
        6+ years of experience as a software engineer. I am also
        working towards a Master's degree in Robotics Engineering
        from Worcester Polytechnic Institute, with a specialization
        in autonomous vehicles.\\

        My current position involves building a semi-autonomous motion system
        for an enormous agricultural machine called an irrigation pivot.
        I improve the system's ability to navigate the field including
        path planning, GPS dead-reckoning, and obstacle detection using
        on-board sensors. I mainly
        do this by writing C/C++ code for a complex embedded system on the machine.
        I write test scripts for the system in Python.\\

        Furthermore, in my Robotics Master's degree program, I'm working with
        a variety of robotics frameworks
        and libraries including ROS2, Gazebo, and TensorFlow. I am developing
        machine learning, deep learning, and reinforcement learning
        for both academic and personal projects. Here, I use Python extensively.\\

        I am also a Co-founder of a very early-stage company where I use
        PyTorch, OpenCV, and YOLOv7 to help daycare workers keep track of
        children that might have been left unattended. In this position,
        I write Python and JavaScript.\\

        Other notable and related experience includes my work with biometric
        facial recognition systems, which relied on a pipeline of computer
        vision and machine learning algorithms. Also, at a previous company,
        I have contributed to the 3 degree of freedom (DOF) motion planning
        of a robotic arm that used a LiDAR sensor to estimate the volume
        of grain in a silo.\\
    \end{itemize}
    
    \item \textbf{Question 2:} What Specific Skills do you posses that make you a strong candidate for this role?
    \begin{itemize}
        \item \textbf{Answer:} I have a strong background in software engineering
        and robotics. I am able to write C, C++, and Python code proficiently
        to develop software for various applications.\\

        I am comfortable with designing and implementing 
        machine learning models, 
        neural networks,
        and reinforcement learning algorithms. I can solve vision
        and path planning problems using OpenCV and ROS2. Because of
        my Master's program, I am familiar with SLAM (Simultaneous
        Localization and Mapping) techniques. Because of my experience 
        in embedded systems, I have
        a strong understanding of sensors and sensor fusion.\\

        I am well-versed in using Git for version control, writing
        requirements and design documents, and writing unit tests for my code.
        I also mentor junior engineers and interns, and I am comfortable
        communicating with both technical and non-technical stakeholders.
    \end{itemize}
    
    \item \textbf{Question 3:} How do you approach problem solving or handling challenges in the workplace?
    \begin{itemize}
        \item \textbf{Answer:} I approach problem solving by first
        understanding the problem and its context. For example, if I am tasked
        with debugging a GPS dead-reckoning algorithm, I would first
        understand how the algorithm is being used on the system as a whole.
        I would collect information about the problem by asking what the
        unexpected and expected behaviors are. Then, I would formulate a hypothesis about the problem and
        test it. If the hypothesis is incorrect, I use what I've learned
        in testing the system so far to formulate a new hypothesis, and
        then repeat the process.\\

        However, sometimes the problem is not well understood, and the
        challenge is to understand the problem itself. In this case,
        I would use a similar approach, but I would also do research
        to understand the problem better. Sometimes this can involve
        studying a technology or sub-field of engineering that I am
        not familiar with. However, when I am comfortable with a certain
        level of understanding, I re-assess the problem and formulate
        a hypothesis on where to look next.
    \end{itemize}
    
    \item \textbf{Question 4:} Can you provide an example of a project or accomplishment you are proud of?
    \begin{itemize}
        \item \textbf{Answer:} I am proud of my work on the irrigation pivot project
        where I built a semi-autonomous motion system for an enormous agricultural
        machine. I collaborated with other engineers to build many parts of this
        system. I wrote drivers for sensors and the network stack, so that 
        information about the system could be communicated to the control
        system. I heavily improved the motion planning algorithm, which
        involved writing a path planner that can intelligently decide
        when to energize the
        motors to move the machine. This takes into account checking
        relays, checking the state of the machine, and checking the
        state of the motors. A mapping of the field is also used to
        determine the best path to take.
        Furthermore, I also wrote a GPS dead-reckoning algorithm
        that can estimate the position of the machine when GPS is not available.
    \end{itemize}
    
    \item \textbf{Question 5:} How do you stay updated and continually improve your knowledge and skills in your field?
    \begin{itemize}
        \item \textbf{Answer:} I have several ways of keeping my knowledge current. First of all,
        I am currently pursuing a Master's degree in Robotics Engineering, so taking new courses
        is one way I continually improve my knowledge and skills. However, my 
        desire to learn new things extends beyond the degree. I have
        taken Coursera courses on machine learning and deep learning, and I
        have also taken Udemy courses on YOLOv7 and YOLOv8. I also read
        research papers on a variety of topics, including reinforcement learning,
        robot dynamics, and computer vision. My sister is also a PhD student
        in Neuroscience, so I often discuss research papers with her, even
        if they are not directly related to my field.\\
    \end{itemize}
    
    \item \textbf{Question 6:} Describe a situation where you had to work collaboratively as part of a team to achieve a goal.
    \begin{itemize}
        \item \textbf{Answer:} I have a few examples of working collaboratively as part of a team, but I 
        will use the instance where I worked with a team of engineers to build a 3 degree of
        freedom (DOF) robotic arm that uses a LiDAR sensor to estimate the volume of grain in a silo.\\

        I was responsible for
        relaying the point cloud data from the LiDAR sensor
        so that it can be transmitted to the cloud via the cellular network on-board.
        I collaborated with a senior firmware engineer to
        retrieve information from the CAN bus interface for the LiDAR sensor.
        I also helped with the motion planning of the robotic arm where it
        needed to perform a sweep of the silo to estimate the volume of grain. Here,
        the simulation was done with a senior mechanical engineer, so we
        used the path planned by the simulation to implement the motion
        planning algorithm. This involved PWM control of the motors, as well
        as sensing the angles of the motors. We also collaborated with an
        electrical engineer to bring up the PCB for the robotic arm, and
        tested the PCB with the motors and sensors.\\
    \end{itemize}
    
    \item \textbf{Question 7:} How do you prioritize your work and manage your time effectively?
    \begin{itemize}
        \item \textbf{Answer:} For me, the best way to do this is by breaking down
        the task at hand into smaller tasks. I then prioritize these tasks
        in a project management tool like Jira. I also use a Kanban board
        to visualize the tasks that I am working on. For each task,
        I estimate the time it will take to complete it, and I thereby also
        estimate the time it will take to complete the entire project. I also
        plan in time for unexpected tasks that might come up. Once I have
        planned, I schedule focus-time on my calendar to work on the tasks.
    \end{itemize}
    
    \item \textbf{Question 8:} Can you share an experience where you demonstrated leadership skills or took initiative?
    \begin{itemize}
        \item \textbf{Answer:} At my current position, we have recently hired
        a new engineer. The week that he started working with us, my team-lead was swamped
        with work, and this was causing the new-hire to sit idle for his first
        couple of days. The new-hire needed someone to help him get up
        to speed and guide him through some tasks.
        At this point, I took the initiative to mentor the new engineer.
        I had a set of tasks that I was working on, and I delegated some
        of these tasks to the new engineer. I also helped him get set up
        with the tools and software that we use. When I took on this initiative,
        I replaced the time I had allocated for the tasks I delegated
        with time for mentoring the new engineer, so that I could
        help him in an organized and effective manner.
        I also made sure to check in with him
        regularly to see if he needed any help. When he did need help,
        I would pair-program with him to help him get unstuck. I showed him
        many debugging techniques, and I also showed him how to perform
        professional code reviews. This made his first few weeks at our
        company very productive, and as a result, my team lead was
        impressed with both of our work. This way, not only did I finish my tasks
        in the time I had already allocated, but I also brought up a new
        engineer to speed, while creating a rapport with him.
    \end{itemize}
    
    \item \textbf{Question 9:} How do you handle feedback and criticism from others?
    \begin{itemize}
        \item \textbf{Answer:} The way I handle feedback and criticism from others
        is by seeking it deliberately and regularly. This way, I mentally prepare
        myself to receive the feedback, and even something harsh is not
        unexpected. To do this, I also attempt to keep the
        demeanor positive by simply asking, ``How can I improve?'' after
        working with someone on a project.

        In general, I am always looking for ways to improve,
        so I attempt to keep an open mind when receiving feedback. When
        someone provides me with suggestions on how to improve, I take
        their perspective with empathy, and I try to understand why they
        are suggesting the change. I also try to understand the context
        of the feedback, and I try to understand the problem that the
        feedback is trying to solve. I then take the feedback and
        incorporate it into my work.
    \end{itemize}
    
    \item \textbf{Question 10:} What interests you about this position and our organization?
    \begin{itemize}
        \item \textbf{Answer:} It has been my dream to work in the field of autonomous
        machines, and I am very excited about the prospect of working at Tuff Torq.
        In particular, I am interested in the work that Tuff Torq is doing
        with off-road autonomous systems. I think this is a unique and
        exciting opportunity to build something that will have a positive
        impact on the world (usually, autonomous vehicles automatically means
        cars, but in this case, it is off-road and unique).
        I am also interested in the fact that Tuff Torq
        is a well-established company with a long history of innovation
        and excellence. I am excited to be a part of this tradition, and
        I look forward to contributing to the company's success.
    \end{itemize}
    
    \item \textbf{Question 11:} How do you adapt to change and handle ambiguity?
    \begin{itemize}
        \item \textbf{Answer:} Change is the only constant in life, and I have
        plenty of experience adapting to change. Let me provide a concrete
        example. All the companies I have
        worked for have been usually acquired by larger companies, and
        this has often resulted in a change in the company culture. Sometimes,
        there has been ambiguity in the work that needs to be done as well.
        However, I have found that keeping a positive attitude is key in
        these situations. Wearing a smile and having a good sense of humor
        can uplift the morale of the team.
        I have also found that being flexible and 
        accommodating to the new situation is important. For example, if
        a new process is introduced, I try to understand the reason for
        the process, and I try to follow it to the best of my ability.\\
        
        When there is doubt
        in what needs to be done, I have found that taking initiative
        helping the company in any way I can is a good way to get started.
        For example, if there is ambiguity on how we need to organize
        the timeline of a project, it is best to take an initiative and
        schedule a meeting to discuss the timeline (as long as it is fine
        with my manager).
    \end{itemize}
    
    \item \textbf{Question 12:} Describe a situation where you had to resolve a conflict with a colleague or team member.
    \begin{itemize}
        \item \textbf{Answer:} At my previous company, I was responsible for 
        implementing the remote configuration of a device over the cellular
        network. Particularly, we wanted to send a large JSON file over a 
        3G cellular connection using a single MQTT message.
        I was tasked with implementing this feature,
        and found that the JSON file was too large to reliably send as
        a single message. We were going to be using this device in
        locations with spotty cellular coverage, so we needed to find a
        way to break-down the JSON file into smaller chunks, or download
        the file using a separate, persistent TCP connection than the MQTT
        connection.\\

        When I approached the cloud-side engineer with this problem, he
        was adamant that the JSON file should be sent as a whole. He touted that
        the theoretical speed of the cellular network was enough to send
        the JSON file. He mentioned that he had read research papers
        that showed that the capabilities of the cellular network were
        far more than what we needed. However, I had seen some data transmission
        logs, and had seen speeds as low as 1 kbps. I also knew that the
        cellular network in some areas of physical installation of the device
        was just not reliable enough to retrieve the entire JSON file as a single
        MQTT message. When we tested the system in our office, the single
        MQTT message was able to be sent just fine, so it was hard to substantiate
        my claim.\\

        This argument between me and the cloud-side engineer went on for
        a few days, and was hindering progress on the project.
        In order to resolve this issue, I approached an engineering manager and
        asked to organize a field trip to see a physical installation of the device.
        In any case we had been wanting to visit sites where the device was
        being used, so this was a good opportunity to do so.
        The trip was approved, and I took the cloud-side engineer with me.
        We went to the physical installation, where we saw that the cellular
        network was indeed unreliable. We also saw that the JSON file was
        unable to be retrieved as a single MQTT message. Then in the following days,
        designed a way to break-down the JSON file into smaller chunks,
        and implemented the feature. This was a good learning experience for us
        both, and we were able to resolve the conflict.
    \end{itemize}
    
    \item \textbf{Question 13:} Can you provide an example of when you had to meet a tight deadline or work under pressure?
    \begin{itemize}
        \item \textbf{Answer:} In May 2023, I was working on a set of tasks to change the GPS
        data communication method. The plan was to try
        the improvement on a customer's machine in the field at the end of the month.
        However, the customer wanted to use
        the machine for the upcoming irrigation season, so we had a tight window
        to test the functionality before the season started.\\

        When we finally went to the field to test the new functionality, we
        found that the GPS data was not being communicated to the control system.
        If we did not get the issue fixed within the next few days, we would potentially
        not be able to test the functionality until after the irrigation season, which
        was several months away.\\

        At this point, I had to work a few 12 hour days to get the issue resolved. The root
        cause of the issue was that we had set up our GPS simulator incorrectly, such that
        the packet of data that was being sent by the simulator was not the same as the
        packet of data that was being sent by the GPS receiver. When we fixed the issue and
        delivered it, our boss was very happy with our work, and rewarded us with a
        lunch at a local restaurant.
    \end{itemize}
    
    \item \textbf{Question 14:} How do you ensure attention to detail and maintain accuracy in your work?
    \begin{itemize}
        \item \textbf{Answer:} Accuracy is a paradigm that goes hand-in-hand with
        validation and verification. This comes by devising a set of requirements
        for the system, and having a team review and approve these requirements.
        Then, I design the system to meet these requirements, and I have a team
        review and approve the design. Next, the implementation details are then
        connected to test cases that verify that the system meets the requirements.
        Finally, the system is tested in a real-world environment to ensure that it
        meets the requirements.\\

        Attention to detail is a bit more nuanced. Usually, when working toward a requirement,
        it is possible to break down the requirement into smaller tasks. Usually, the more
        fine-grained the task, the more attention to detail is achieved. I always endeavor
        to break down my tasks into the smallest possible tasks, and I even set aside
        some of them as ``nice-to-haves''. If time allows, I will work on these ``nice-to-haves''
        to ensure that my work is as detailed as possible. The final step to
        this is making sure that my code is readable and maintainable. This is done by
        having other engineers review my code, and by writing unit tests for my code.
    \end{itemize}
    
    \item \textbf{Question 15:} What is your preferred work environment and management style?
    \begin{itemize}
        \item \textbf{Answer:} I work best in situations where I have plenty of time to focus
        on my work. I prefer to work in a quiet environment, and I like to have
        a set of tasks that I can work on.\\

        I prefer to work with a manager who is hands-off, but is available to
        answer questions and provide guidance when needed. I also prefer to
        work with a manager who is open to listening to my ideas and suggestions. At the same
        time, I appreciate when I get regular feedback from my manager, and
        recognizes when I have taken steps to act upon the feedback.\\
    \end{itemize}
    
    \item \textbf{Question 16:} Can you describe a time when you had to handle multiple tasks simultaneously and how you manage them?
    \begin{itemize}
        \item \textbf{Answer:} Currently, I have two different projects that I am working on in my
        current position. One of them is coordination of a consulting team that is
        based in India. I am helping them with the development of a new-generation of
        our product. The other project is the development of a new feature for our
        smart-pivot product that a proprietary idea to quickly figure out the alignment
        status of the pivot. For this feature, I am writing the software-level requirements,
        designing the software architecture, and implementing the software.\\

        The way I manage these tasks is by looking at the priority of the next upcoming
        deliverable. If a lot of progress can be made in a short amount of time, I take
        care of that task first. I make sure that on average in a week, I have divided
        my time between the two projects depending on the priority of the deliverables.
        In general, I have found that by planning my expectations for the week, I can
        manage my time effectively.\\
    \end{itemize}
    
    \item \textbf{Question 17:} How do you handle failure or setbacks in your work?
    \begin{itemize}
        \item \textbf{Answer:} Failures are the stepping stones to success. In order
        to turn this adage into a technical approach, I use the concept of ``regression''.
        When I encounter a failure, I first try to understand the root cause of the failure.
        Then, I try to understand why the failure was not caught earlier. I then devise
        a test case that will catch the failure in the future. This is fundamentally
        the concept of regression testing, and is the engineering equivalent of
        learning from your mistakes.\\
    \end{itemize}
    
    \item \textbf{Question 18:} Can you share an example of how you have successfully handled a difficult client or customer?
    \begin{itemize}
        \item \textbf{Answer:} At my first position after college, I worked on improving the
        computer vision systems for a biometric facial recognition system. The system was supposed
        to be integrated into an Indian payments app called PayTM\@. We traveled to India to
        assist the integration of this app. In one of our meetings with the stakeholders of the
        integration project, the client asked to know how exactly our algorithms worked. We explained
        that we used a pipeline of machine learning and computer vision algorithms to detect
        the eye-region of a face, and then we used another pipeline to match the person's known
        information with the detected eye-region. The client was not satisfied with this answer,
        and probed with more questions. He said that in the case of a technology audit, we need
        to know exactly how the algorithms work, with a decision tree of the algorithm.\\

        The research scientists and engineers on my team knew that this was not possible, because
        neural networks were viewed as black-boxes at the time. On top of this, we were not
        allowed to share the source code of our algorithms with the client. However, the client
        was adamant that we provide a detailed explanation of our algorithms.\\

        The client convinced more of his team members that we were not being transparent with
        our algorithms, and this resulted in a lot of pressure on our team. We were also in
        a different country, and delays in our schedule meant higher costs for the company (travel, 
        lodging, etc.).\\

        At this point, I took the initiative to schedule a meeting with the client to discuss
        the issue directly with the product manager on our team. I explained the blocking issue
        to our product manager, who had recently flown in from the US\@. The product manager
        saw how this was an issue that would cause a lot of delays, and he decided to take
        the issue to the CEO of our company. The CEO then had a meeting with the CEO of PayTM\@.
        In this meeting, the CEO of our company explained the issue with the client's request,
        and the CEO of PayTM understood the issue. This eventually led to contractual changes
        that allowed us to continue with the project.\\

        This was a situation which could have caused a lot of delays and costs for the company,
        but by taking the initiative to resolve the issue, I was able to help the company
        avoid these costs. I also learned that sometimes, it is important to take the initiative
        to resolve an issue, even if it is not directly related to my work.
    \end{itemize}
    
    \item \textbf{Question 19:} What are your long term career goals, and how does this position align with them?
    \begin{itemize}
        \item \textbf{Answer:} My ultimate career goal is to create autonomous systems that
        can help people in their daily lives, by giving them more time to do the things they
        love. So much of our time is taken up by doing mundane tasks, and I believe that
        we are happier when we have more time to spend time with our loved ones or to
        pursue our passions. Furthermore, I am particularly interested in the field of autonomous vehicles.\\
        
        This position at Tuff Torq combines both of these goals. Tuff Torq is working on
        autonomous systems that can help people in their daily lives.
        I'm very excited about the prospect of working at Tuff Torq, and I
        look forward to contributing to the company's success.
    \end{itemize}
    
    \item \textbf{Question 20:} Do you have any questions for us about the position or the company?
    \begin{itemize}
        \item \textbf{Answer:} Yes, I do have a few questions.
        \begin{itemize}
            \item What is the company culture like at Tuff Torq?
            \item To build autonomous systems, many teams are usually involved. What
            teams would I be working with?
            \item What is the typical career path for someone in this position?
            \item What do the engineers at Tuff Torq like most about working there?
            \item What are the new and exciting projects that Tuff Torq is working on?
        \end{itemize}
    \end{itemize}
\end{itemize}

\end{document}
